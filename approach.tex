\startcomponent approach
\environment report-environment

\startchapter[title=Approach]
\startsection[title=Dataset]
We kept the same sentence “Bob Dylan is a songwriter” that the authors of “Language models are Open Knowledge Graphs” used as an input. The proposed method is unsupervised since it doesn't require any training process or data. 
It only requires a single application of BERT per paragraph (instead of applying BERT for every possible triplet), which is relatively efficient.
But mainly, the idea of looking into the pooled attention weights for linking entities and relations is novel.
\stopsection

\startsection[title=SILKNOW Knowledge Graph]
SILKNOW is a research project based on records from existing catalogs that aims to produce digital modeling of weaving techniques (a “Virtual Loom”), through automatic visual recognition, advanced spatio-temporal visualization, multilingual and semantically enriched access to digital data.
\blank
SILKNOW improves the understanding, conservation and dissemination of European
silk heritage from the 15th to the 19th century.
By applying next-generation computing research to the needs of diverse users (museums, education, tourism, creative industries, media\dots), it preserves the tangible and intangible heritage associated with silk
\blank
The description of each silk heritage is based on controlled vocabularies that are essentials to link entities.
The API access is developed for KG and the exploratory search engine ADASilk is created on top of it.
Automatic image and text analysis are applied to predict missing metadata in the KG.
\blank
As for OpenNRE and FRED, we used four sentences from SILKNOW Knowledge Graph (full sentences in Annex section).
\stopsection
\stopchapter

\stopcomponent